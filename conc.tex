\section{Conclusion}
\label{sec:conc}

In this paper, we focused on analyzing and comparing the performance of three algorithms - an Exponential Information Gathering protocol for consensus \cite{KM13} (algorithm EIG), a randomized byzantine agreement protocol which uses the concept of \textit{quorums} to filter requests \cite{BPV06} (algorithm Quorum), and a fast byzantine agreement algorithm that uses the Pull-Push model of communication \cite{BGH13} (algorithm Pull-Push). We considered three metrics for comparison: (1) \textit{communication} complexity, (2) \textit{round} complexity, and (3) \textit{time} complexity. 

We showed that in general the algorithm Pull-Push performed better than the other two in terms of communication complexity. For increasing network sizes the communication complexity remains the same if the increase is less than double. But, this performance comes with a trade-off. When time complexity was considered, as can be seen in Fig. \ref{fig:elapsed16}, algorithm Quorum performed better. This comes with the assumption that certain rounds or sub-protocols in algorithm Quorum could be executed in parallel. However, this good performance of Quorum was not reflective in its communication complexity, and it showed poor performance on comparison with algorithm Pull-Push. In real-time situations, as the number of processes in a network increase the probability of having faulty nodes in the network naturally increases. Hence, even though algorithm EIG shows better performance when the fault ratio is small, one should keep in mind if there is a possibility of high ratio of faults in the future.

