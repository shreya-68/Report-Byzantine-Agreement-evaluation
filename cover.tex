\begin{center}

\textbf{Submission to Information Sciences Special Issue on Dependability in Parallel and Distributed Systems and Applications}
\end{center}


\noindent Dear Guest Editors, \\

We would like to submit the following manuscript entitled
``A Performance Comparison of Algorithms for Byzantine
Agreement in Distributed Systems'' for publication review for the 
Special Issue on Dependability in Parallel and Distributed Systems and Applications.

The Byzantine Agreement problem aims to achieve reliability in consensus
protocols despite the failure of some number of components.  
Analysis of algorithms under varying parameters and practical constraints through experimental evaluation can be key to understanding the performance and trade-offs of theoretically well-performing algorithms.

We focus on implementation and analysis of three performant Byzantine
Agreement algorithms with best results for their respective objectives of
optimizing only communication complexity or latency, or meeting the
worst-case bounds on both.

Our experimental analysis shows that the algorithms perform differently for different network sizes.
For networks with strict bandwidth constraints, algorithm by
Braud-Santoni et al.~\cite{BGH13} allows high flexibility in changing the size
of the network. For high variance of fault ratios, algorithm by Ben-Or et al.
\cite{BPV06} prevented byzantine processes from influencing the communication
complexity. Comparison of elapsed real time and CPU time utilization provides
better insight into the utility of one of the parallel algorithms.  Moreover,
we make modifications to the algorithm by Kowalski et al. \cite{KM13} to
improve the communication bits sent in every round of the protocol.

Quantifying the performance of the algorithms empirically provides a practical
understanding and insight into how the different algorithms perform under different
conditions to achieve consensus in distributed systems.

Thank you. \\

\noindent Sincerely,

\noindent Shreya Agrawal and Khuzaima Daudjee 

\noindent University of Waterloo

%%%%%%%%%%%%%%%%%%%%%%%%%%%%%%%%%%%%%
%We focus on implementation and analysis of three recently proposed algorithms with best algorithms for their respective agendas---(1) algorithm \textit{Quorum} by Ben-Or et al. \cite{BPV06}, (2) algorithm \textit{Pull-Push} by Braud-Santoni et al.~\cite{BGH13}, and (3) algorithm \textit{EIG} (Exponential Information Gathering) by Kowalski et al. \cite{KM13}. 
%In general, the randomized algorithm \textit{Pull-Push} performs better than
%the other two in terms of communication complexity. When latency is considered,
%algorithm \textit{Quorum} performs better.
%
%In the case of a small network, with the number of processes less than $32$ and
%a low fault percentage---(approximately less than $10\%$), algorithm
%\textit{EIG} performs better since it considers the number of faults in its
%protocol. However, its performance degrades under high fault ratio. We show
%that under high variance of network size over time if the increase in size is
%within the next power of $2$, then for algorithm \textit{Pull-Push} the number
%of bits sent per node remains the same. For networks with strict bandwidth
%constraints, this allows high flexibility in changing the size of the
%network. For high variance of fault ratios, algorithm \textit{Quorum} prevented
%byzantine processes from influencing the communication complexity. Moreover,
%we improve upon the results obtained for algorithm \textit{EIG} by making
%certain changes in the algorithm itself---removing the redundant information
%being sent in every round regarding the list of byzantine nodes. This further
%%improves the number of bits sent in every round of the protocol.
