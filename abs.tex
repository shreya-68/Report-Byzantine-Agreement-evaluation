\title{A System Performance Comparison of Byzantine Agreement in Distributed Systems}
%\subtitle{CS654 Project Proposal}


%\author{
%\IEEEauthorblockN{Shreya Agrawal}
%\IEEEauthorblockA{School of Computer Science\\
%University of Waterloo\\
%s8agrawa@uwaterloo.ca }
%
%\and
%\IEEEauthorblockN{Khuzaima Daudjee}
%\IEEEauthorblockA{School of Computer Science\\
%University of Waterloo\\
%kdaudjee@cs.uwaterloo.ca }}

\maketitle

\begin{abstract}

Ensuring that all processes in a network agree on all decisions in the
presence of byzantine processes is known to be a challenging task in
distributed applications. This is fundamentally due to two reasons: (1)
asynchronous execution of systems, and (2) a high ratio of byzantine
nodes colluding together to inject arbitrary values in a synchronous
system. These issues pose a problem to scalability, safety, liveness,
reliability and performance of the system. Impossibility results have
been shown for deterministic solutions for the first case, even in the
presence of one fault. For the second case, many researchers have taken
different approaches to find better performing solutions. While strong
theoretical results give us an insight into the efficiency of an
algorithm, they sometimes come with large hidden constants and might not
be practical for a real system. In this paper, we compare the relative
performance of two randomized algorithms, one using the {\em pull-push} approach
and the other using the concept of {\em quorums}, and a third recent
deterministic algorithm to show that simple algorithms can perform
better in certain cases. We show this by implementing a testbed
environment. Three metrics have been used for comparison - {\em bit}
complexity, {\em time} complexity and {\em round} complexity with respect to varying network sizes and
faulty processes. Using these, we identify instances when each of these
algorithms perform differently under varying resource constraints and
number of faults. We show that for small networks ($n<32$) and upto $10\%$ of faulty nodes, a simple deterministic algorithm performs better than any randomized one. However, for large networks, a recently proposed randomized algorithm performs better than any other state-of-the-art algorithm. Although the trade-off here is between communication and time complexity, the second parallel randomized algorithm performs much better in terms of round complexity than the other two. 

\end{abstract}

\noindent {\bf Keywords:} Distributed systems, Performance, Byzantine failures, Fault-tolerant, Consensus, Complexity.
