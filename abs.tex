\title{A Performance Comparison of Algorithms for Byzantine Agreement in Distributed Systems}
%\subtitle{CS654 Project Proposal}


%\author{
%\IEEEauthorblockN{Shreya Agrawal}
%\IEEEauthorblockA{School of Computer Science\\
%University of Waterloo\\
%s8agrawa@uwaterloo.ca }
%
%\and
%\IEEEauthorblockN{Khuzaima Daudjee}
%\IEEEauthorblockA{School of Computer Science\\
%University of Waterloo\\
%kdaudjee@cs.uwaterloo.ca }}

\maketitle

\begin{abstract}

Ensuring that all processes in a network agree in the
presence of byzantine processes is known to be a challenging task in
distributed systems. This is fundamentally due to two reasons: (1)
asynchronous execution of systems, and (2) byzantine
processes colluding together to inject arbitrary values in a synchronous/asynchronous
system. These issues pose a problem to scalability, safety, liveness,
reliability and performance of the system. Impossibility results have
been shown for deterministic solutions for the first case, even in the
presence of one fault. For the second case, researchers have taken
different approaches to find performant solutions. While strong
theoretical results give us an insight into the efficiency of an
algorithm, they sometimes come with large hidden constants and might not
be practical for a real system. In this paper, we compare the 
performance of two randomized algorithms, one using the {\em pull-push} approach
and the other using the concept of {\em quorums}, and a third recent simple
deterministic algorithm that we expect to perform
better in certain cases.
Three metrics have been used for comparison - {\em bit}
complexity, {\em round} complexity and {\em latency} with respect to varying network sizes and
faulty processes. By implementing a testbed environment and using these metrics, we identify instances for which each 
algorithm performs differently under varying resource constraints and
number of faults. We show that for small networks ($n<32$) and up to $10\%$ of faulty processes, the simple deterministic algorithm performs better than any of the randomized ones. However, for larger networks, the best performance is shown by the algorithm with the pull-push approach. Although, the trade-off here is between communication and round complexity, the second randomized algorithm performs much better in terms of round complexity than the other two. 

\end{abstract}

\noindent {\bf Keywords:} Distributed systems, Performance, Byzantine failures, Fault-tolerant, Consensus, Complexity.
