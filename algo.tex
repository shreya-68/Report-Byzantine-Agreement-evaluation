\section{Algorithms}
\label{sec:algos}

\begin{table*}[t]
    \begin{tabular}{p{2cm}llllllp{2.5cm}p{2cm}}
\hline
\textbf{Algorithm} & \textbf{Year} & \textbf{Type} &\textbf{n} & \textbf{Rounds} & \textbf{Bit Complexity } &\textbf{Decision value} & \textbf{Communicating nodes}& \textbf{Remarks} \\ \hline
Ben-Or, Pavlov, Vaikuntanathan \cite{BPV06} & 2006 & Randomized & $4k + 1$ & $O(logn)$ & $n^{O(logn)}$      &String of $O(logn)$ bits &All-to-all communication and within quorums of size $O(logn)$   & Everywhere BA     \\
          Braud-Santoni et. al \cite{BGH13} & 2014 & Randomized & $3k + 1$ & $O(\frac{logn}{log\; log n})$ & $\tilde{o}(n)$      &String of $O(logn)$ bits &With samplers of size $O(logn)$  & Almost-everywhere to everywhere     \\
 Kowalski and Mostefaoui \cite{MHR14} & 2013   & Deterministic & $3k + 1$     & $k + 1$               &Single bit               & $O(n^3logn)$ &All-to-all communication & Uses EIG data structure\\ \hline
%
\end{tabular}
\end{table*}

%TODO: Define the model: Network, Adversary, synch. Defn of basic byzantine agreement and any other notations.

%TODO: Describe the three algorithms in detail --> a page each (include their basic func, theoretical results, analysis) what you want to evaluate

Here, we will describe the model of the network, adversarial conditions and the algorithms of \cite{BPV06}, \cite{BGH13} and \cite{MHR14}, i.e., algorithm \textit{Quorum}, \textit{Pull-Push} and \textit{EIG}, respectively, that are to be evaluated under these conditions. Table 1 compares some properties of these algorithms.


\subsection{Model}
Our model is a \textit{fully-connected} network of $n$ processes, with authenticated communication channels - the identity of the sender is known to the recipient and authentication is not required during the execution of the protocol. We require the network to be \textit{reliable}, i.e., a message sent (to a non-faulty node) will be eventually delivered and in order. We consider only the \textit{synchronous} model of communication. In this setting, the communication proceeds in rounds and all processes have synchronized clocks; a process moves on to the next round after all the processes have completed the previous round. %messages are all sent out at the same time at the start of the round, and then received at the same time at the end of the same round. 

\subsection{Adversary}
The adversary controls a fraction of the processes - a maximum of $k$ processes, which are called \textit{byzantine} processes. Such processes deviate arbitrarily from the algorithm by crash failures or sending false messages. The assumption is that the adversary is \textit{non-adaptive} for all the three algorithms, that is the adversary chooses the set of byzantine processes at the start of the protocol. The adversary is computationally unbounded and has \textit{full information} about the state of all processes and the network, and the communication between any pair of them. Another characteristic is that the adversary is \textit{rushing}. This means that the adversary knows all the messages sent by the good processes in all the previous rounds and the current round before choosing which messages to send in the current round. This is also known as the \textit{full information} model \cite{GGL98}.

\subsection{Input string}
Both the algorithms Quorum and Pull-Push  make use of a string of bits as the final decision. However, algorithm EIG uses a single bit as input and output decision value.


\subsection{Algorithms\protect\footnote{The algorithms presented here have been reproduced, paraphrased, condensed from works of \cite{BPV06}, \cite{KM13} and \cite{BGH13}.}}

\subsubsection{Algorithm Quorum\cite{BPV06}}
The main result of this algorithm, due to Ben-Or et al., is as follows:
\begin{theorem}
For any constant $\epsilon > 0$, there exists a protocol $\mathcal{P}$ that reaches Byzantine Agreement in a synchronous full-information network tolerating $k < (1/4 - \epsilon) n$ non-adaptive Byzantine faults, and runs for expected $O(logn/\epsilon^2)$ rounds. 
\end{theorem}
The algorithm makes use of a weaker version of broadcast known as \textit{Graded Broadcast} \cite{FM97}, with a designated authority called the `dealer'.
%\begin{itemize}
%\item If the dealer is good, all the processes get the same message
%\item Even if the dealer is bad, if some good player accepts the message, all the good processes get the same message (but they may or may not accept it)
%\end{itemize}
\begin{definition}
A protocol $P$ is said to achieve graded broadcast if, at the beginning of the protocol the dealer $D$ holds a value $v$, and at the end of the protocol, every process $P_i$ outputs a pair $(v_i, c_i)$ where $c_i \in \{0, 1, 2\}$ denotes the confidence of the process in value $v_i$. With that, the following properties should hold:
\begin{enumerate}
\item If $D$ is honest, then $v_i = v$ and $c_i = 2$ for every honest process $P_i$. 
\item For any two honest processes $P_i$ and $P_j$, $\mid c_i - c_j \mid \leq 1$.
\item (Consistency) For any two honest processes $P_i$ and $P_j$, if $c_i > 0$ and $c_j > 0$, then $v_i = v_j$.
\end{enumerate}
\end{definition}

The protocol is described in detail as follows: \\
\textbf{Input to the Dealer $D$:} A value $v$  \\
\textbf{Output of process $P_i$:} A pair $(v_i, c_i)$ \\
\textbf{Step 1} The dealer $D$ distributes $v$ to all the processes. \\
\textbf{Step 2} (For every process $P_i$) Send $v_i$, the received value from the dealer, to all other processes.  \\
\textbf{Step 3} (For every process $P_j$) Let $v_i^j$ denote the message from process $P_i$ in Step 2. If there is a value $\mu$ such that $\geq n - k$ of the $v_i^j$'s are equal to $\mu$, then send $\mu$ to all the processes. Else, send $\bot$. \\
\textbf{Step 4} (For every process $P_i$) Let $\mathtt{num}_\mu$ denote the number of players that sent $\mu$ to $P_i$ in Step 3. \\
\begin{itemize}
\item If $\mathtt{num}_\mu \geq 2k + 1$ for some $\mu$, output $(\mu, 2)$.
\item If $2k \geq \mathtt{num}_\mu \geq k + 1$ for some $\mu$, output $(\mu, 1)$.
\item If $\mathtt{num}_\mu \leq k $ for all $\mu$, output $(\bot, 0)$.
\end{itemize}


        \begin{minipage}[h]{0.90\columnwidth}
        %\footnotesize
        \begin{algorithm}[H]
         \caption{Stage 1: Form Committees}%
               \label{alg:stage1}%
      %\SetAlgoLined
               \KwIn{Process $P_i$ with a random number $B_i$}
        \KwOut{View of all committees formed by all the processes}

        $B_i \gets rand([1, \dots, \frac{n}{alogn}])$\;

        $gradecast(B_i)$\;

        
        \ForEach{ $P_j$ in $P$}{

        $(v_i^j, c_i^j) \gets recv(P_j)$\;

        }

        \ForEach{ $C_k$ in $C$}{

            \ForEach{$P_j$ in $P$}{
                \If{$v_i^j == C_k$ and $c_i^j == 2$}{

                    $\mathtt{view}_i(C_k) \gets \mathtt{view}_i(C_k) \cup P_j$\;
                }
            }
        }
        \ForEach{ $C_k$ in $C$}{

            \ForEach{$P_j$ in $P$} {

                \uIf{$v_i^j == C_k$ and $c_i^j > 2$}{

                    $\mathtt{\bar{view}}_i(C_k) \gets \mathtt{\bar{view}}_i(C_k) \cup P_j$\;
                }
            }
        }

    \end{algorithm}%
  \end{minipage}%


This algorithm for Byzantine Agreement proceeds in stages. It makes use of Feige's protocol \cite{Feige99} for collective coin-flipping which works as follows: In the first round, all the processes throw a ball at random into one of $O(n/log\;n)$ bins. The processes which throw their ball into the \textit{lightest bin} survive. The protocol is invoked recursively on the $O(log\;n)$ processes in the lightest bin. Algorithm \ref{alg:stage1} and \ref{alg:stage2} detail this procedure of agreeing on a small committee.


  \begin{minipage}[h]{0.90\columnwidth}%
    \footnotesize%
        \begingroup
        \removelatexerror
            \begin{algorithm}[H]%
         \caption{Stage 2: Agree on smallest committee}%
               \label{alg:stage2}%
        \KwIn{View of all committees formed by all the processes}
        \KwOut{Smallest Committee}


        \ForEach{ $C_i \in C$}{
            thread\_create$(t_k, \mathtt{decide}(C_i))$\;
        }
        
        decide$(C_i)$ \;
            \Indp $S \gets \mathtt{subsets}(\frac{3}{4}a\mathtt{log n}, P)$\;

            \ForEach{ $S_j \in S$}{
                thread\_create$(t_{S_j}, \mathtt{subprotocol}(S_j, C_i))$\;
            }
        subprotocol$(S_j, C_i)$ \;

                \ForEach{$P_k \in S_j$}{
                    $\mathtt{\bar{view}}_k(C_i) \gets \mathtt{\bar{obtain\_view}}(P_k, C_i$) \;
                    $\mathtt{\bar{view}}_{S_j}(C_i) \gets \mathtt{\bar{view}}_{S_j}(C_i) \cap \mathtt{\bar{view}}_k(C_i)$\;
                }
                \eIf{$\mid \mathtt{\bar{view}}_{S_j}(C_i) \mid > \mathtt{alog n}$}{
                    $\mathtt{disq}_{P_i, S_j}(C_i) = 1$\;
                }{
                    $\mathtt{disq}_{P_i, S_j}(C_i) = 0$\;
                }
                \If{$\mathtt{disq}_{P_i, S_j}(C_i) == 0$}{
                    \ForEach{$P_k \in S_j$}{
                        $\mathtt{view}_k(C_i) \gets \mathtt{obtain\_view}(P_k, C_i$) \;
                        $\mathtt{composition}_{S_j}(C_i) \gets \mathtt{composition}_{S_j}(C_i) \cap \mathtt{view}_k(C_i)$\;
                    }
                }
                broadcast$(S_j, C_i, \mathtt{disq}_{P_i, S_j}(C_i), \hfill \mathtt{composition}_{S_j}(C_i))$\;

                \tcc{The tuple is received from ID $S_j'$}
                $(S_j, C_i, \mathtt{disq}, \mathtt{comp}) \gets \mathtt{recv}(S_j')$\;

                \ForAll{$(S_j, C_i, 1, *)$}{
                    \If{$\mid S_j' \mid \geq \frac{1}{2}\mathtt{alogn}$ and $S_j' \subseteq S_j \subseteq \mathtt{view}_i(C_i)$}{
                        $\mathtt{disq}_{P_i}(C_i) = 1$\;
                    }
                }
                \ForEach{$S_j \in \mathtt{view}_i(C_i)$}{
                    \ForAll{$S_j', (S_j, C_i, 0, D) $}{
                        \If{$ (S_j' \subseteq S_j)$ and $(\mid S_j' \mid \geq \frac{1}{2}\mathtt{alogn})$}{
                            $\mathtt{composition}_{S_j}(C_i) \gets D$\;
                        }
                        {
                            $\mathtt{composition}_{S_j}(C_i) \gets \bot$\;
                        }
                    }
                    \ForAll{$S_j$}{
                            $\mathtt{composition}(C_i) \gets$ largest$(\mathtt{composition}_{S_j}(C_i))$\;
                    }
                }
            \Indm \ForAll{$C_i$}{
                 $C_{small} \gets$ Choose\_smallest($\mathtt{composition}(C_i)$)\;
            }
    \end{algorithm}%
    \endgroup
  \end{minipage}%
%}


The key idea is that honest processes are unbiased and the resulting bin will contain a large fraction of honest processes. After $log^* n$ invocations of the process, a leader is elected. The leader then flips a coin, and broadcasts it, which is the agreed value. The challenge is that dishonest processes will exhibit byzantine behavior when throwing their ball into one of the bins. This is overcome by using the \textit{gradecast} protocol described earlier. Algorithm \ref{alg:stage3} describes the leader election procedure.



%\textbf{Stage 1: Form $\frac{n}{alogn}$ Committees}\\


%\textbf{Stage 2: Agree on smallest committee}\\

%    \fbox{
%\begin{multicols}{2}

%\newcommand{\part}[1]{%

%\textbf{Stage 3: Final Decision}\\
%    \fbox{
        \begin{minipage}[h]{0.90\columnwidth}
            \begin{algorithm}[H]
    %\footnotesize
         \caption{Stage 3: Final Decision}%
               \label{alg:stage3}%
      %\SetAlgoLined
               \KwIn{Smallest committee chosen by all processes $C_{small}$}
        \KwOut{Final decision value of all processes}

    \tcc{$L_i$ is the ID of leader of committee $C_{small}$}
    $L_i \gets$ leader\_election$(C_{small})$\;
    \eIf{$L_i == P_i$}{
        val = flip\_coin()\;
        broadcast(val)\;
    }{
        val = recv()\;
    }
    \Return{val}



    \end{algorithm}%
  \end{minipage}%
%  }




\subsubsection{Algorithm Pull-Push \cite{BGH13}}
This algorithm by Braud-Santoni et al. \cite{BGH13} is the first probabilistic Byzantine Agreement algorithm whose communication and time complexities are poly-logarithmic. The authors use an \textit{almost-everywhere} algorithm of \cite{KSSV06} and extend the protocol to Byzantine agreement in the complete network. This protocol uses the \textit{pull-push} communication model. The following theorem shows the main result: 


\begin{theorem}
For $n$ nodes in a synchronous full-information message passing model with a non-adaptive non-rushing Byzantine adversary which controls less than a $1/3 - \epsilon$ fraction of the nodes, if more than $3/4$ of the nodes know a string $g_{\mathtt{string}}$ (random enough), there is an algorithm such that with high probability:
\begin{itemize}
\item At the end of the algorithm, each correct node knows $g_{\mathtt{string}}$.
\item The algorithm takes $O(1)$ rounds and $\tilde{O}(n)$ messages are exchanged in total. 
\end{itemize}
\end{theorem}

The algorithm makes use of \textit{quorums} to filter requests or messages sent by other processes. The choice of quorums each node uses is directed by both deterministically-known information (like the identity of the node), and random sources (randomly chosen initially string). Such quorums are called \textit{Samplers}. The reason for doing so is as follows:
\begin{itemize}
\item If the nodes choose deterministically who they contact, either there will be a linear number of them or few enough for the adversary to corrupt a majority. 
\item If uniformly random quorums are chosen, the byzantine nodes might follow the algorithm but increase the worst-case communication complexity by contacting many disjoint quorums. 
\end{itemize}

Each process starts with a candidate string (the string to be agreed upon). The assumption is that more than half of the processes are both correct and have the same candidate string. The algorithm proceeds in two phases. In each of the phases, messages are sent only to selectively chosen processes or are received from selectively chosen processes by sampler functions. The detailed algorithm is given as follows:


\textbf{Push Phase}

 In the first phase, called the push phase, each process starts to \textit{diffuse} (send) its candidate string $g_{\mathtt{string}}$. Initially, node $x$'s candidate list $L_x$ contains $g_{\mathtt{string}}$. A push occurs when a node receives information about their candidate string from other nodes without asking for it. To each pair of string $s$ and node $x$, the sampler $I$ assigns a set of $O(log\;n)$ nodes. $x$ may receive pushes for $s$ from nodes in $I(x,s)$: the push Quorum of $x$ according to $s$. If more than half of the nodes of $I(x,s)$ push for $s$, $s$ is added to $x$'s candidate list $L_x$. Refer to Algorithm \ref{alg:push} for the detailed implementation.



        \begin{minipage}[h]{0.90\columnwidth}
        \begingroup
        \removelatexerror
        \begin{algorithm}[H]
         \caption{Push phase}%
         \footnotesize
               \label{alg:push}%
      %\SetAlgoLined
               \KwIn{Process $P_i$ with a random string $g_{string}$, a list of all strings $C_{string}$}
        \KwOut{Each node creates a candidate strings list $L_{P_i}$}
        
        $g_{string} \gets$ createRandString()\;

        broadcast$(g_{string})$\;

        $L_{P_i} \gets g_{string}$\;

        
        \ForEach{ $str \in C_{string}$}{

            $I_{str} \gets$ getPushQuorum($\mathtt{str}, P_i$)\;
            \ForEach{$p \in I_{str}$}{
                \If{$recv(I_s) == str$}{
                    $num++$\;
                }
            }
            \If{$\mathtt{num} > len(I_s)$}{
                $L_{P_i} \gets L_{P_i} \cup s$\;
            }

        }


    \end{algorithm}%
    \endgroup
  \end{minipage}%


\begin{minipage}[H]{0.90\columnwidth}
    \begin{Ualgorithm}
         \footnotesize
         \caption{Sending Pull Request}%
               \label{alg:send_pull}%
      %\SetAlgoLined
               \KwIn{$L_{P_i}$, list of candidates for node $P_i$}
        \KwOut{String agreed upon}
        
        
        \ForEach{ $s \in L_{P_i}$}{
            $r_{P_i, s} \gets$ generateRand()\;
            $J_{r,s} \gets$ getPollList($r_{P_i, s}, P_i$)\;
            $H_s \gets$ getPullQuorum($s$, $P_i$)\;
            send$(POLL, s, r_{P_i, s}, J_{r,s})$\;
            send$(PULL, s, r_{P_i, s}, H_s)$\;

            Upon event: recv$(ANSWER,s,r)  \Leftarrow w$\;
            \Indp 
                \If{$w \in J_{r,s}$}{
                    $\mathtt{count}_s ++$\;
                    \If{$\mathtt{count}_s > \frac{1}{2}|J_{r,s}|$}{
                        has\_decided $\gets$ true\;
                        $\mathtt{final}_s \gets s$\;
                        \Return{$s$}\;
                    }
                }
        }
    \end{Ualgorithm}%
  \end{minipage}%




        \begin{minipage}[H]{0.90\columnwidth}
        \begingroup
        \removelatexerror
        \begin{algorithm}[H]
         \caption{Routing Pull Request}%
    \footnotesize
               \label{alg:rout_pull}%
      %\SetAlgoLined
        Upon event: $\mathtt{recv}(PULL, s, r_{x, s}, H_s) \Leftarrow x$\;
        \Indp 
            \If{$(g_{string} == s)$ and $(P_i \in H_s)$}{
                $J_{x,r_{x,s}} \gets$ getPollList($r_{x, s}, x$)\;
                \ForEach{$w \in J_{x,r_{x,s}}$}{
                    $H_{w,s} \gets$ getPullQuorum($s$, $w$)\;
                    send$(ROUTE, x, s, r_{x, s}, w) \Rightarrow H_{w,s}$\;
                }
            }
        \Indm Upon event: $\mathtt{recv}(ROUTE, x, s, r_{x,s}, w) \Leftarrow P_j$\;
        \Indp 
            $H_{x,s} \gets$ getPullQuorum($s$, $x$)\;
            $J_{x,r_{x,s}} \gets$ getPollList($r_{x, s}, x$)\;
            \If{$(g_{string} == s)$ and $(P_j \in H_{x,s})$ and $(w \in J_{x,r_{x,s}})$}{
                $\mathtt{fw\_count}_{s,x}++$\;
                \If{$\mathtt{fw\_count}_{s,x} > \frac{1}{2}|H_{x,s}|$}{
                    send$(FORWARD, x, s, r_{x, s}) \Rightarrow w$\;
                    $\mathtt{fw\_count}_{s,x} \gets \infty$\;
                }
            }


    \end{algorithm}%
    \endgroup
  \end{minipage}%




\textbf{Pull Phase}

In the second phase, called the pull phase, the bogus strings are discarded so that each process keeps only the correct string. This is done by each process requesting the strings it received in the push phase to be verified by some other processes. A \textit{pull} query is sent out to receive information about each string as a consequence. Checking a string $s$ involves a Poll List $J(x, r_x)$ and a Pull Quorum $H(x, s)$, where $r_x$ is taken at random. 


Algorithms \ref{alg:send_pull}, \ref{alg:rout_pull} and \ref{alg:ans_pull} give a detailed implementation of the sending, routing and answering of pull requests.
%\begin{enumerate}
%\item Sending Queries: Each node $x$ verifies each string $s \in L_x$ by polling a set of nodes. A random string $r_{x,s}$ is chosen to define $J(x, r_{x,s})$. A different random string is used for each candidate string $s$. Next, $x$ sends a `POLL' request to Poll List $J(x, r_{x,s})$ and a `PULL' request to Push Quorum $H(x, s)$.
%\item Answering: 
%\begin{enumerate}
%\item A node $y \in H(x, s)$ forwards a request received from $x$ iff $s$ is its initial candidate string $s_y$. The request is forwarded to nodes in $J(x, r_x)$ routed through their Pull Quorums by sending a `ROUTE' request.
%\item A node $z$ in the Pull Quorum of $w \in J(x, R_x)$ ($z \in H(s, w)$) forwards the request to $w$ iff $s = s_z$ and $z$ received the request from more than half of the nodes of $H(x, s)$. 
%\item Finally, a node $w \in J(x, r_x)$ replies to a `PULL' request from $x$ if:
%\begin{itemize}
%\item the pull request was received from a majority of $H(w, s)$;
%\item either one of its pull requests was answered (thus $w$ knows $g_{\mathtt{string}}$), and $s_w$ was changed accordingly;
%\item or it currently has received less than $log^2n$ pull requests. 
%\end{itemize}
%\end{enumerate}
%\item Deciding: If $x$ receives answers from a majority of nodes in $J(x, r_{x,s})$, $s$ is deemed to be the global string.
%\end{enumerate}




        \begin{minipage}[H]{0.90\columnwidth}
        \begingroup
        \removelatexerror
        \begin{algorithm}[H]
         \footnotesize
         \caption{Answering Pull Request}%
               \label{alg:ans_pull}%
      %\SetAlgoLined
        Upon event: $\mathtt{recv}(ROUTE, x, s, r_{x, s}) \Leftarrow z$\;
        \Indp 
            \If{$\mathtt{count}_s > \mathtt{log}^2n$}{
                Wait for has\_decided\;
            }
            $J_{x,r_{x,s}} \gets$ getPollList($r_{x, s}, x$)\;
            $H_{P_i,s} \gets$ getPullQuorum($s$, $P_i$)\;
            \If{$(g_{string} == s)$ and $(P_i \in J_{x, r_{x,s}})$ and $(z \in H(P_i, s))$}{
                $\mathtt{fw\_count}_{s,x}++$\;
                \If{$(\mathtt{fw\_count}_{s,x} > \frac{1}{2}|H_{x,s}|)$ and $((x, s) \in \mathtt{Polled})$}{
                    $\mathtt{count}_s ++$\;
                    send$(ANSWER, s, r_{x, s}) \Rightarrow x$\;
                }
            }
        \Indm Upon event: $\mathtt{recv}(POLL, s, r_{x,s}) \Leftarrow x$\;
        \Indp 
            $J_{x,r_{x,s}} \gets$ getPollList($r_{x, s}, x$)\;
            \If{$P_i \in J_{x, r_{x,s}}$}{
                $\mathtt{Polled} \gets \mathtt{Polled} \cup (x,s)$\;
                \If{$\mathtt{fw\_count}_{s,x} > \frac{1}{2}|H_{x,s}|$}{
                    $\mathtt{count}_s ++$\;
                    send$(ANSWER, s, r_{x, s}) \Rightarrow x$\;
                    $\mathtt{fw\_count}_{s,x} \gets \infty$\;
                }
            }

    \end{algorithm}%
    \endgroup
  \end{minipage}%





\subsubsection{Algorithm EIG \cite{KM13}}
This protocol, similar to the classic protocol by \cite{Bar-NoyD91}, has two phases. In the first phase, each process communicates with every other process for $k + 1$ rounds and stores the collected information in each round at a corresponding level in a tree-like data structure. In the second phase, a bottom-up evaluation is done on each of the trees at each process. The fundamental difference between this protocol and past EIG solutions is that after a couple of rounds instead of storing and sending proposed values at each level of the tree, only an array of suspected byzantine nodes is sent. This array is updated after each round using what is known as the confirmation mechanism which works as follows:
\begin{itemize}
\item A process $p_i$ sends \textit{main information}, which could be an array of values at round $r$ and every other process receives this information in the same round.
\item At round $r+1$, process $p_j$ \textit{echoes} information received from $p_i$ received in previous round to every other process. 
\item Each process $p_j$ receives the echoed information from every other process. If the main information received from $p_i$ in round $r$ is echoed by at least $(n - k)$ processes in round $r+1$, then process $p_j$ is said to \textit{confirm} the information received from $p_i$ in round $r$.
    \end{itemize}

    The following is the description of the rounds $1$ to $k+1$ of the protocol: \\
    \textbf{Round 1} Each process sends its proposed value to all other processes. Received values in this round are used to decorate nodes at level $1$ of the EIG tree. $\mathtt{val(x)}$ is set to the received value $v$ from node $x$. \\
    \textbf{Round 2} Each process echoes the messages received in the first round and received messages decorate the second level. For a node $x = jk$, $\mathtt{cval(x)} = v$ where $v$ is the value process $p_k$ reports that it received from $p_j$ in previous round. The confirmation mechanism is applied to the received echoed messages by each process $p_i$ and whichever node's main information is not confirmed in this round it is added to $p_i$'s byzantine list. \\
    \textbf{Round 3} The echoed messages received in round $2$ are sent as echo messages and the suspected byzantine list is sent as the main information. Confirmation mechanism is applied to the echoed information and byzantine list is updated. \\
    \textbf{Round r, ($4 \leq r \leq k+1$)} From 4th round onwards the suspected byzantine list of a process is sent as its main information and byzantine lists received in the previous round are echoed. Confirmation mechanism is applied to the received echo information and byzantine lists are once again updated. Nodes at corresponding levels of the EIG tree for round 3 onwards, for each process $p_i$, are given values for $\mathtt{val}$ and $\mathtt{cval}$ as follows:
    \begin{itemize}
        \item For node $x=ykl$, $\mathtt{val(x)} = \top$, if $p_l$ never reported to $p_i$ by round $k$ that it suspects $p_k$, otherwise $\mathtt{val(x)} = \bot$.
        \item For node $x=yjkl$, $\mathtt{cval(x)} = \top$, if $p_l$ never reported to $p_i$ by round $k+1$ that $p_k$ suspects $p_j$, otherwise $\mathtt{cval(x)} = \bot$.
        \item For leaf nodes $x = ykl$, $\mathtt{val(x)} = \top$, if $p_l$ did not report to $p_i$ that it suspects $p_k$, otherwise $\mathtt{val(x)} = \bot$.
    \end{itemize}
where $y$ is a string of ids (possibly empty), and $j, k, l$ are ids of three different processes. The important point to note here is that in every round even though the number of nodes in the EIG tree increases by factor $n$, the new information received remains quadratic which allows us to use arrays to store this information. Even the size of the echoed messages remains quadratic in every round. 

\textbf{Extracting the final information:}\\
Starting from the leaves, the nodes of the EIG tree are evaluated bottom-up as follows:
\begin{itemize}
    \item if $x$ is a leaf, $\mathtt{newval}(x) \leftarrow \mathtt{val}(x)$
    \item if $x$ is root, $\mathtt{newval}(x) \leftarrow v$ such that strict majority of new values of its children are set to $v$, otherwise it is set to default value $v_0$.
    \item otherwise, $\mathtt{newval}(x) \leftarrow v$ if for $T = \{y \mid y \text{ child of } x \wedge \mathtt{newval}(y) = \top\}$, $\mid T \mid \geq (n-t-l)$, and there is a strict majority of $\mathtt{cval}$ of nodes in $T$ that are set to $v$.
\end{itemize}


%The first algorithm to make use of an Exponential Information Gathering (EIG) protocol was given by Bar-Noy et al. \cite{Bar-NoyD91}. The algorithm was shown to have a lower bound of $k + 1$ rounds for time complexity, where $k$ is the number of faulty processes such that $k < n/3$ \cite{FischerLP83}. In 1998, Garay and Moses \cite{GarayM98}, with modifications to the two phase protocol of \cite{Bar-NoyD91} using the EIG data structure, improved the communication complexity further to polynomial time. \\
%% The communication complexity was exponential in the number of faulty processes $f$ which was later improved to $\Omega(n^2)$ in \cite{DolevR85}.
%
%
%We describe the EIG protocol due to Bar-Noy et al. as follows. In the first round of the protocol, each process broadcasts its initial value $v_i$ to all other processes. In each of the following $t$ rounds, every process broadcasts all of the information it received in the latest round. At the end of the $t+1$ rounds, each process computes a decision value based on the information it has gathered, decides on this value and halts. The data structure used, called the EIG tree, is described as follows. The tree has a depth of at most $t+1$, with a node at depth $r$ has $n-r$ children. The edges of the tree are labelled with process names. The outgoing edges of the root are labelled with $1, \dots, n$ respectively. A node of depth $r \geq 1$ has an edge labelled $i$ for every process name $i$ that does not appear on the path leading from the root to the node. Each node is associated with two values - $\mathtt{val}(x)$ and $\mathtt{newval}(x)$, for a node $x$. $\mathtt{val}(x)$ is assigned during the $t+1$ rounds of information exchange. Initially, the root holds the input value of that node. In each round $k$, the node broadcasts information stored at depth $k-1$ of the tree. If process $j$ reports a value $v$ for node $\sigma$ to node $i$, then $\mathtt{val}(\sigma j) = v$ in $i$'s tree. 
%
%In the last \textit{recursive majority voting} phase, starting from the leaves up to the node, each node in process's tree, is assigned $\mathtt{newval}(x)$ to be the majority of $\mathtt{val}$ values of among all of node $x$'s children. The final value decided by a process $i$ is the value of $\mathtt{newval}(root)$ in process $i$'s tree.
%
%
%\subsubsection{Summary of complexities}
%
%A summary of the time and communication complexities of the results mentioned above has been summarized below:
%
%\begin{center}
%    \begin{table}[h]
%    \begin{tabular}{| l | l | l | l |}
%        \hline
%        Protocol & n & time & communication \\ \hline
%        Pease et al. \cite{PeaseSL80} & $3f + 1$ & $f+1$ & exp($n$) \\ \hline
%         Garay et al. \cite{GarayM98} & $3f+1$ & $f+1$ & poly($n$) \\ \hline
%      Ben-Or et al. \cite{BPV06} & $4f + 1$ & $O(logn)$ & $n^{O(logn)}$ \\ \hline
%        Braud-Santoni et al. \cite{BGH13} & $3f + 1$ & Polylog & Polylog \\ \hline
%    \end{tabular}
%\end{table}
%\end{center}
%
