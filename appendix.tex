\section{Appendix}
\label{sec:appendix}

\subsection{Algorithm Quorum\cite{BPV06}}

\begin{definition}
A protocol $P$ is said to achieve graded broadcast if, at the beginning of the protocol the dealer $D$ holds a value $v$, and at the end of the protocol, every process $P_i$ outputs a pair $(v_i, c_i)$ where $c_i \in \{0, 1, 2\}$ denotes the confidence of the process in value $v_i$. With that, the following properties should hold:
\vspace{-2mm}
\begin{enumerate}
\item If $D$ is honest, then $v_i = v$ and $c_i = 2$ for every honest process $P_i$. 
\item For any two honest processes $P_i$ and $P_j$, $\mid c_i - c_j \mid \leq 1$.
\item (Consistency) For any two honest processes $P_i$ and $P_j$, if $c_i > 0$ and $c_j > 0$, then $v_i = v_j$.
\end{enumerate}
\end{definition}

The graded broadcast algorithm is described in detail as follows: \\
\textbf{Input to the Dealer $D$:} A value $v$  \\
\textbf{Output of process $P_i$:} A pair $(v_i, c_i)$ \\
\textbf{Step 1} The dealer $D$ distributes $v$ to all the processes. \\
\textbf{Step 2} (For every process $P_i$) Send $v_i$, the received value from the dealer, to all other processes.  \\
\textbf{Step 3} (For every process $P_j$) Let $v_i^j$ denote the message from process $P_i$ in Step 2. If there is a value $\mu$ such that $\geq n - k$ of the $v_i^j$'s are equal to $\mu$, then send $\mu$ to all the processes. Else, send $\bot$. \\
\textbf{Step 4} (For every process $P_i$) Let $\mathtt{num}_\mu$ denote the number of players that sent $\mu$ to $P_i$ in Step 3. \\
\vspace{-4mm}
\begin{itemize}
\item If $\mathtt{num}_\mu \geq 2k + 1$ for some $\mu$, output $(\mu, 2)$.
\item If $2k \geq \mathtt{num}_\mu \geq k + 1$ for some $\mu$, output $(\mu, 1)$.
\item If $\mathtt{num}_\mu \leq k $ for all $\mu$, output $(\bot, 0)$.
\end{itemize}

\subsection{Algorithm Pull-Push \cite{BGH13}}



        \begin{minipage}[H]{0.80\columnwidth}
        \begin{Ualgorithm}
         \caption{Push phase}%
               \label{alg:push}%
      %\SetAlgoLined
               \KwIn{Process $P_i$ with a random string $g_{string}$, a list of all strings $C_{string}$}
        \KwOut{Each node creates a candidate strings list $L_{P_i}$}
        
        $g_{string} \gets$ createRandString()\;

        broadcast$(g_{string})$\;

        $L_{P_i} \gets g_{string}$\;

        
        \ForEach{ $str$ in $C_{string}$}{

            $I_s \gets$ getPushQuorum($\mathtt{str}, P_i$)\;
            $\mathtt{num} \gets recv(I_s)$\;
            \If{$\mathtt{num} > len(I_s)$}{
                $L_{P_i} \gets L_{P_i} \cup s$\;
            }

        }


    \end{Ualgorithm}%
  \end{minipage}%



        \begin{minipage}[H]{0.80\columnwidth}
        \begin{Ualgorithm}
         \caption{Sending Pull Request}%
               \label{alg:send_pull}%
      %\SetAlgoLined
               \KwIn{$L_{P_i}$, list of candidates for node $P_i$}
        \KwOut{String agreed upon}
        
        
        \ForEach{ $s$ in $L_{P_i}$}{
            $r_{P_i, s} \gets$ generateRand()\;
            $J_{r,s} \gets$ getPollList($r_{P_i, s}, P_i$)\;
            $H_s \gets$ getPullQuorum($s$, $P_i$)\;
            send$(POLL, s, r_{P_i, s}, J_{r,s})$\;
            send$(PULL, s, r_{P_i, s}, H_s)$\;

            Upon event: recv$(ANSWER,s,r)  \Leftarrow w$\;
            \Indp 
                \If{$w$ in $J_{r,s}$}{
                    $\mathtt{count}_s ++$\;
                    \If{$\mathtt{count}_s > \frac{1}{2}|J_{r,s}|$}{
                        has\_decided $\gets$ true\;
                        $\mathtt{final}_s \gets s$\;
                        \Return{$s$}\;
                    }
                }
        }


    \end{Ualgorithm}%
  \end{minipage}%




        \begin{minipage}[H]{0.90\columnwidth}
        \begingroup
        \removelatexerror
        \begin{algorithm}[H]
         \caption{Routing Pull Request}%
    \footnotesize
               \label{alg:rout_pull}%
      %\SetAlgoLined
        Upon event: $\mathtt{recv}(PULL, s, r_{x, s}, H_s) \Leftarrow x$\;
        \Indp 
            \If{$(g_{string} == s)$ and $(P_i \in H_s)$}{
                $J_{x,r_{x,s}} \gets$ getPollList($r_{x, s}, x$)\;
                \ForEach{$w \in J_{x,r_{x,s}}$}{
                    $H_{w,s} \gets$ getPullQuorum($s$, $w$)\;
                    send$(ROUTE, x, s, r_{x, s}, w) \Rightarrow H_{w,s}$\;
                }
            }
        \Indm Upon event: $\mathtt{recv}(ROUTE, x, s, r_{x,s}, w) \Leftarrow P_j$\;
        \Indp 
            $H_{x,s} \gets$ getPullQuorum($s$, $x$)\;
            $J_{x,r_{x,s}} \gets$ getPollList($r_{x, s}, x$)\;
            \If{$(g_{string} == s)$ and $(P_j \in H_{x,s})$ and $(w \in J_{x,r_{x,s}})$}{
                $\mathtt{fw\_count}_{s,x}++$\;
                \If{$\mathtt{fw\_count}_{s,x} > \frac{1}{2}|H_{x,s}|$}{
                    send$(FORWARD, x, s, r_{x, s}) \Rightarrow w$\;
                    $\mathtt{fw\_count}_{s,x} \gets \infty$\;
                }
            }


    \end{algorithm}%
    \endgroup
  \end{minipage}%





        \begin{minipage}[H]{0.80\columnwidth}
        \begin{Ualgorithm}
         \footnotesize
         \caption{Answering Pull Request}%
               \label{alg:ans_pull}%
      %\SetAlgoLined
        Upon event: $\mathtt{recv}(ROUTE, x, s, r_{x, s}) \Leftarrow z$\;
        \Indp 
            \If{$\mathtt{count}_s > \mathtt{log}^2n$}{
                Wait for has\_decided\;
            }
            $J_{x,r_{x,s}} \gets$ getPollList($r_{x, s}, x$)\;
            $H_{P_i,s} \gets$ getPullQuorum($s$, $P_i$)\;
            \If{$g_{string} == s$ and $P_i$ in $J_{x, r_{x,s}}$ and $z$ in $H(P_i, s)$}{
                $\mathtt{fw\_count}_{s,x}++$\;
                \If{$\mathtt{fw\_count}_{s,x} > \frac{1}{2}|H_{x,s}|$ and $(x, s)$ in $\mathtt{Polled}$}{
                    $\mathtt{count}_s ++$\;
                    send$(ANSWER, s, r_{x, s}) \Rightarrow x$\;
                }
            }
        \Indm Upon event: $\mathtt{recv}(POLL, s, r_{x,s}) \Leftarrow x$\;
        \Indp 
            $J_{x,r_{x,s}} \gets$ getPollList($r_{x, s}, x$)\;
            \If{$P_i$ in $J_{x, r_{x,s}}$}{
                $\mathtt{Polled} \gets \mathtt{Polled} \cup (x,s)$\;
                \If{$\mathtt{fw\_count}_{s,x} > \frac{1}{2}|H_{x,s}|$}{
                    $\mathtt{count}_s ++$\;
                    send$(ANSWER, s, r_{x, s}) \Rightarrow x$\;
                    $\mathtt{fw\_count}_{s,x} \gets \infty$\;
                }
            }

    \end{Ualgorithm}%
  \end{minipage}%


%\begin{enumerate}
%\item Sending Queries: Each node $x$ verifies each string $s \in L_x$ by polling a set of nodes. A random string $r_{x,s}$ is chosen to define $J(x, r_{x,s})$. A different random string is used for each candidate string $s$. Next, $x$ sends a `POLL' request to Poll List $J(x, r_{x,s})$ and a `PULL' request to Push Quorum $H(x, s)$.
%\item Answering: 
%\begin{enumerate}
%\item A node $y \in H(x, s)$ forwards a request received from $x$ iff $s$ is its initial candidate string $s_y$. The request is forwarded to nodes in $J(x, r_x)$ routed through their Pull Quorums by sending a `ROUTE' request.
%\item A node $z$ in the Pull Quorum of $w \in J(x, R_x)$ ($z \in H(s, w)$) forwards the request to $w$ iff $s = s_z$ and $z$ received the request from more than half of the nodes of $H(x, s)$. 
%\item Finally, a node $w \in J(x, r_x)$ replies to a `PULL' request from $x$ if:
%\begin{itemize}
%\item the pull request was received from a majority of $H(w, s)$;
%\item either one of its pull requests was answered (thus $w$ knows $g_{\mathtt{string}}$), and $s_w$ was changed accordingly;
%\item or it currently has received less than $log^2n$ pull requests. 
%\end{itemize}
%\end{enumerate}
%\item Deciding: If $x$ receives answers from a majority of nodes in $J(x, r_{x,s})$, $s$ is deemed to be the global string.
%\end{enumerate}

